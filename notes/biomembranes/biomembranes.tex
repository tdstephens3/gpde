% biomembranes.tex August 2016

\documentclass[10pt]{article}
\usepackage{color}
\usepackage{graphicx}
\usepackage{customMath}
\setlength{\parindent}{0pt}
\usepackage[top=1.25cm, bottom=2cm, left=2.0cm, right=1.5cm]{geometry} 

\pdfpagewidth 8.5in
\pdfpageheight 11in
%

\title{Biomembranes}
\author{Tom Stephens}
\date{ \today}

%%%%%%%%%%%%
\begin{document}         %%% <----------  BEGIN DOCUMENT
%%%%%%%%%%%%
\maketitle



% SECTION: 
\section{Differential Geometry for Surface PDE}

Codimension-1 surfaces of dimension $d$ embedded in ($d+1$)-dimensional
Euclidean space can be described through parameterizations $\chi: \R^d \to
\R^{d+1}$, or through the level set of a function $b: \R^{d+1} \to \R$.  In
either case the differential geometry of these surfaces should be expressed in
terms of the simple formulas of the ambient Euclidean metric space, but this
requires some work.  The problem is that, unless the surface is completely
`flat', any distance along the surface must be translated into a distance in
the Euclidean metric space, and vice versa. To express the change in a function
value with respect to a change in $x$, one must disambiguate the space in which
$x$ is allowed to move. If $x$ ranges through the familiar Euclidean metric
space, then in general it will cover a greater distance along the surface.
Likewise, if $x$ ranges through a distance along the surface, then in general
that distance will appear shorter in the direction it moves in Euclidean space.
These discrepancies must be accounted for in order to compute derivatives of
non-constant quantities on the surface.  



\subsection{Differential Geometry for Parametric Surfaces, $\chi : (\xi_1, \ldots,
   \xi_d) \mapsto \gamma \subset \R^{d+1}$}

Let $\gamma$ be a smooth, closed codimension-one manifold in $\R^3$ 

For $V \subset \R^2$ with $\xi_1,\xi_2 \in V$, put  $\chi : (\xi_1,\xi_2)
\mapsto \left( \chi_1(\xi_1,\xi_2), \chi_2(\xi_1,\xi_2), \chi_3(\xi_1,\xi_2)
\right) = x \in \gamma$

Denote the tangent space at $x \in \gamma$ by $T_x {\gamma} := \text{span}
\{ \lambda^1, \lambda^2\}$ where $\lambda_1,\lambda^2$ are linearly independent
vectors tangent to $\gamma$ at $x$.

Write the Jacobian of $\chi$ as $D\chi(\xi_1,\xi_2) = \left( \partial_1 \chi,
\partial_2 \chi \right) := \left(
\begin{array}{cc}
  \ds  \frac{\partial \chi_1}{\partial \xi_1} & \ds \frac{\partial \chi_1}{\partial \xi_2} \\ 
                                          &                                        \\
  \ds  \frac{\partial \chi_2}{\partial \xi_1} & \ds \frac{\partial \chi_2}{\partial \xi_2} \\ 
                                          &                                        \\
  \ds  \frac{\partial \chi_3}{\partial \xi_1} & \ds \frac{\partial \chi_3}{\partial \xi_2} 
\end{array} \right)$ 

\begin{definition}[Metric Tensor $g_{ij}: x \mapsto \R^{2x2}$, and First Fundamental Form I $: (x;\,\lambda^1, \lambda^2) \in T_x \gamma \to \R$] 
   
\end{definition}



\begin{definition}[Curvature Tensor $h_{ij}: x \mapsto \R^{2x2}$, and Second Fundamental Form II $: (x; \, \lambda^1, \lambda^2) \in T_x \gamma \to \R$] 
   
\end{definition}


unit normal vector $n$, $g = g_{ij}$, $h = h_{ij}$, Laplace-Beltrami operator $\Delta_{\gamma}$,
surface gradient $\nabla_{\gamma}$


\begin{definition}[Surface Gradient] Let $\chi : V \to \gamma$ be
   a smooth parameterization of the surface $\gamma$ over $V \subset \R^2$. For
   $v : \gamma \to \R$  a smooth function, the 
   \emph{surface gradient} $\nabla_{\gamma} v = \nabla_\gamma
   v(\chi(\xi_1,\xi_2))$ is defined as 

   \begin{align}
      \nabla_\gamma v(\chi(\xi_1,\xi_2)) 
      :&= D\chi \, g^{-1} \,\nabla (v \circ \chi)(\xi_1,\xi_2) \\
      &= \left( 
         \begin{array}{cc}
           \frac{\partial \chi_1}{\partial \xi_1} &  \frac{\partial \chi_1}{\partial \xi_2} \\ 
           \frac{\partial \chi_2}{\partial \xi_1} &  \frac{\partial \chi_2}{\partial \xi_2} \\ 
           \frac{\partial \chi_3}{\partial \xi_1} &  \frac{\partial \chi_3}{\partial \xi_2} 
         \end{array} \right)
         \left(
         \begin{array}{cc}    
            g^{11} & g^{12} \\
            g^{21} & g^{22} 
         \end{array} \right)
         \left(
         \begin{array}{c}    
            \frac{\partial v}{\partial \xi_1} \\
            \frac{\partial v}{\partial \xi_2}
         \end{array} \right)
      \label{eqn:surface_grad}
   \end{align}

   Notice that this entire computation is made in $V \subset \R^2$.  
   
\end{definition}


\begin{definition}[Tangential Divergence]
   
\end{definition}

\begin{definition}[Surface Laplacian, the Laplace Beltrami operator] Let $\chi : V \to \gamma$ be
   a smooth parameterization of the surface $\gamma$ over $V \subset \R^2$. For
   $v : \gamma \to \R$  a smooth function, the 
   \emph{surface Laplacian} $\Delta_{\gamma} v = \Delta_\gamma v(\chi(\xi_1,\xi_2))$ is defined as 

   \begin{equation}
      \Delta_\gamma v(\chi(\xi_1,\xi_2)) := \emph{div}_{\gamma}\, \left( (\nabla_\gamma
      v)^T \right)
      \label{eqn:surface_laplacian}
   \end{equation}
   
\end{definition}


% thm: vector mean curvature
\begin{theorem}[Vector Mean Curvature]{ $Hn = - \Delta_{\gamma}\, \emph{id}_\gamma$ }
   \label{thm:vector_mean_curvature}
\end{theorem}

\begin{proof}
\end{proof}

\subsection{Differential Geometry for Surfaces Defined through Level Sets,
   $\gamma := \set{x \in \R^{d+1} \st b(x) = 0}$}



% SECTION: Evolution Equations on Stationary Surfaces
\section{Evolution Equations on Stationary Surfaces}


% subsection: Heat Equation
\subsection{Heat Equation}

\begin{equation}
   u_t = \kappa \Delta_{\gamma} u
   \label{eqn:heat}
\end{equation}

{\bf Weak form:} 

{\bf Discretization:} the $\theta$-method: 
\begin{equation}
   \langle \phi_h, \, \frac{u_h^{n+1} - u_h^n}{\tau} \rangle = \langle \phi_h,
   \, \left[ (1 - \theta)\Delta_\gamma u_h^{n} + \theta
   \Delta_{\gamma} u_h^{n+1}\right]\rangle
   \label{disc:heat_theta}
\end{equation}




% subsection: Allen-Cahn 
\subsection{Allen-Cahn}

\begin{equation}
u_t = \eps \Delta_{\gamma} u + \lambda (u - u^3)
   \label{eqn:heat}
\end{equation}




% SECTION: Evoling Surfaces
\section{Evolving Surfaces}

The classical Helfrich model (\ref{eqn:helfrich_energy}) approximates the energy
required to hold a phospholipid bilayer in a geometric configuration that
deviates from its preferred, or spontaneuous, curvature. The model also
includes energy penalties $\lambda$ and
$\mu$ for changes in surface area and enclosed volume. Treating $\lambda$ and
$\mu$ as Lagrange multipliers capture the fact that surface area and enclosed
volume changes happen on a longer timescale than conformational changes.\\



For $\Gamma \subset \R^3$, a closed smooth codimension-one surface
\begin{equation}
   E({\gamma}) := \kappa(x) \int_{\gamma} \left( H(x) - c_{0}(x)\right)^2  + \lambda + \mu\, s
\cdot n \, dS 
   \label{eqn:helfrich_energy}
\end{equation}
where $H(x) = k_1(x) + k_2(x)$: sum of \emph{principle curvatures} $k_1$ and
$k_2$, $c_0(x)$: \emph{spontaneous curvature}, and $\kappa(x)$: \emph{bending
modulus}.


\cite{DoganNochetto:2012}


% SECTION: Evolution Equations on Evoling Surfaces
\section{Evolution Equations on Evolving Surfaces}



% SECTION: Numerics
\section{Numerics}

\subsection{Scalar Finite Elements for Geometric PDE and Evolving Parametric Surfaces}

\subsection{Vector Finite Elements for Geometric PDE and Evolving Parametric Surfaces}

% EXAMPLE: numerical approximation of vector mean curvature
\begin{example}[Weakly Enforcing an Identity: computing the vector mean curvature
   on a parametric surface]

{\bf Numerical Approximation of vector mean curvature $Hn$:}

\[\langle \phi, \, H n \rangle = \langle \nabla_\gamma \phi, \, \nabla_\gamma\,
\text{id}_\gamma \rangle \]

The matrix equation $MH = rhs$, where $rhs_{i} = \sum_{K \in \mcal{T}} \int_{K} \nabla_{\gamma}
\phi_i : \nabla_{\gamma}\, \text{id}_\gamma \,dx$, $M_{ij} = \sum_{K \in
   \mcal{T}} \int_{K} \phi_i : \phi_j \, dx$  
   
\end{example}

\subsection{Schur Complement}

\subsection{adaptive time stepping}

\subsection{adaptive mesh refinement/coarsening}





\bibliographystyle{plain}
\bibliography{../../../vedas/archives/references/runningBib}


\end{document}
%EOF
