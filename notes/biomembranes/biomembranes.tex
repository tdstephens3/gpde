% biomembranes.tex August 2016

\documentclass[10pt]{article}
\usepackage{color}
\usepackage{graphicx}
\usepackage{customMath}
\setlength{\parindent}{0pt}
\usepackage[top=1.25cm, bottom=2cm, left=2.0cm, right=1.5cm]{geometry} 

\pdfpagewidth 8.5in
\pdfpageheight 11in
%

\title{Biomembranes}
\author{Tom Stephens}
\date{ \today}

%%%%%%%%%%%%
\begin{document}         %%% <----------  BEGIN DOCUMENT
%%%%%%%%%%%%
\maketitle



% SECTION: 
\section{Differential Geometry for Surface PDE}

Let $\gamma$ be a smooth, closed codimension-one manifold in $\R^3$ 

For $U \subset \R^2$ with $\xi_1,\xi_2 \in U$, put  $\chi : (\xi_1,\xi_2)
\mapsto \left( \chi_1(\xi_1,\xi_2), \chi_2(\xi_1,\xi_2), \chi_3(\xi_1,\xi_2)
\right) \in \gamma$


Write the Jacobian of $\chi$ as $D\chi(\xi_1,\xi_2) = \left( \partial_1 \chi,
\partial_2 \chi \right) := \left(
\begin{array}{cc}
  \ds  \frac{\partial \chi_1}{\partial \xi_1} & \ds \frac{\partial \chi_1}{\partial \xi_2} \\ 
                                          &                                        \\
  \ds  \frac{\partial \chi_2}{\partial \xi_1} & \ds \frac{\partial \chi_2}{\partial \xi_2} \\ 
                                          &                                        \\
  \ds  \frac{\partial \chi_3}{\partial \xi_1} & \ds \frac{\partial \chi_3}{\partial \xi_2} 
\end{array} \right)$ 

\begin{definition}[Metric Tensor $g_{ij}$, and First Fundamental Form I $: (\lambda^1,
   \lambda^2) \to \R$] 
   
\end{definition}



\begin{definition}[Curvature Tensor $h_{ij}$, and Second Fundamental Form II $: (\lambda^1,
   \lambda^2) \to \R$] 
   
\end{definition}


unit normal vector $n$, $g = g_{ij}$, $h = h_{ij}$, Laplace-Beltrami operator $\Delta_{\gamma}$,
surface gradient $\nabla_{\gamma}$


\begin{definition}[Surface Gradient] Let $\chi : U \to \gamma$ be
   a smooth parameterization of the surface $\gamma$ over $U \subset \R^2$. For
   $v : \gamma \to \R$  a smooth function, the 
   \emph{surface gradient} $\nabla_{\gamma} v = \nabla_\gamma
   v(\chi(\xi_1,\xi_2))$ is defined as 

   \begin{equation}
      \nabla_\gamma v(\chi(\xi_1,\xi_2)) := D\chi \, g^{-1} \,\nabla (v \circ
      \chi)(\xi_1,\xi_2)
      \label{eqn:surface_grad}
   \end{equation}
   
\end{definition}


\begin{definition}[Tangential Divergence]
   
\end{definition}

\begin{definition}[Surface Laplacian, the Laplace Beltrami operator] Let $\chi : U \to \gamma$ be
   a smooth parameterization of the surface $\gamma$ over $U \subset \R^2$. For
   $v : \gamma \to \R$  a smooth function, the 
   \emph{surface Laplacian} $\Delta_{\gamma} v = \Delta_\gamma v(\chi(\xi_1,\xi_2))$ is defined as 

   \begin{equation}
      \Delta_\gamma v(\chi(\xi_1,\xi_2)) := \emph{div}_{\gamma}\, \left( (\nabla_\gamma
      v)^T \right)
      \label{eqn:surface_laplacian}
   \end{equation}
   
\end{definition}


% thm: vector mean curvature
\begin{theorem}[Vector Mean Curvature]{ $Hn = \Delta_{\gamma}\, \emph{id}_\gamma$ }
   \label{thm:vector_mean_curvature}
\end{theorem}

\begin{proof}
\end{proof}




% SECTION: Evolution Equations on Stationary Surfaces
\section{Evolution Equations on Stationary Surfaces}


% subsection: Heat Equation
\subsection{Heat Equation}

\begin{equation}
   u_t = \kappa \Delta_{\gamma} u
   \label{eqn:heat}
\end{equation}

{\bf Weak form:} 

{\bf Discretization:} the $\theta$-method: 
\begin{equation}
   \langle \phi_h, \, \frac{u_h^{n+1} - u_h^n}{\tau} \rangle = \langle \phi_h,
   \, \left[ (1 - \theta)\Delta_\gamma u_h^{n} + \theta
   \Delta_{\gamma} u_h^{n+1}\right]\rangle
   \label{disc:heat_theta}
\end{equation}




% subsection: Allen-Cahn 
\subsection{Allen-Cahn}

\begin{equation}
u_t = \eps \Delta_{\gamma} u + \lambda (u - u^3)
   \label{eqn:heat}
\end{equation}




% SECTION: Evoling Surfaces
\section{Evolving Surfaces}

The classical Helfrich model (\ref{eqn:helfrich_energy}) approximates the energy
required to hold a phospholipid bilayer in a geometric configuration that
deviates from its preferred, or spontaneuous, curvature. The model also
includes energy penalties $\lambda$ and
$\mu$ for changes in surface area and enclosed volume. Treating $\lambda$ and
$\mu$ as Lagrange multipliers capture the fact that surface area and enclosed
volume changes happen on a longer timescale than conformational changes.\\



For $\Gamma \subset \R^3$, a closed smooth codimension-one surface
\begin{equation}
   E({\gamma}) := \kappa(x) \int_{\gamma} \left( H(x) - c_{0}(x)\right)^2  + \lambda + \mu\, s
\cdot n \, dS 
   \label{eqn:helfrich_energy}
\end{equation}
where $H(x) = k_1(x) + k_2(x)$: sum of \emph{principle curvatures} $k_1$ and
$k_2$, $c_0(x)$: \emph{spontaneous curvature}, and $\kappa(x)$: \emph{bending
modulus}.


\cite{DoganNochetto:2012}


% SECTION: Evolution Equations on Evoling Surfaces
\section{Evolution Equations on Evolving Surfaces}



% SECTION: Numerics
\section{Numerics}

\subsection{Scalar Finite Elements for Geometric PDE and Evolving Parametric Surfaces}

\subsection{Vector Finite Elements for Geometric PDE and Evolving Parametric Surfaces}

% EXAMPLE: numerical approximation of vector mean curvature
\begin{example}[Weakly Enforcing an Identity: computing the vector mean curvature
   on a parametric surface]

{\bf Numerical Approximation of vector mean curvature $Hn$:}

\[\langle \phi, \, H n \rangle = \langle \nabla_\gamma \phi, \, \nabla_\gamma\,
\text{id}_\gamma \rangle \]

The matrix equation $MH = rhs$, where $rhs_{i} = \sum_{K \in \mcal{T}} \int_{K} \nabla_{\gamma}
\phi_i : \nabla_{\gamma}\, \text{id}_\gamma \,dx$, $M_{ij} = \sum_{K \in
   \mcal{T}} \int_{K} \phi_i : \phi_j \, dx$  
   
\end{example}

\subsection{Schur Complement}

\subsection{adaptive time stepping}

\subsection{adaptive mesh refinement/coarsening}





\bibliographystyle{plain}
\bibliography{../../../vedas/archives/references/runningBib}


\end{document}
%EOF
