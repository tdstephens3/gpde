% biomembranes.tex August 2016

\documentclass[10pt]{article}
\usepackage{color}
\usepackage{graphicx}
\usepackage{customMath}
\setlength{\parindent}{0pt}
\usepackage[top=1.25cm, bottom=2cm, left=2.0cm, right=1.5cm]{geometry} 

\pdfpagewidth 8.5in
\pdfpageheight 11in
%

\title{Biomembranes}
\author{Tom Stephens}
\date{ \today}

%%%%%%%%%%%%
\begin{document}         %%% <----------  BEGIN DOCUMENT
%%%%%%%%%%%%
\maketitle



% SECTION: 
\section{Differential Geometry for Surface PDE}

Let $\gamma$ be a smooth, closed codimension-one manifold in $\R^3$ 

parameterized by $\chi : U \subset \R^2 \to \gamma$

unit normal vector $n$, $g = g_{ij}$, $h = h_{ij}$, Laplace-Beltrami operator $\Delta_{\gamma}$,
surface gradient $\nabla_{\gamma}$

% thm: vector mean curvature
\begin{theorem}[Vector Mean Curvature]{ $Hn = \Delta\, \emph{id}_\gamma$ }
   \label{thm:vector_mean_curvature}
\end{theorem}

\begin{proof}
\end{proof}

% Numerical Approximation of vector mean curvature
{\bf Numerical Approximation of vector mean curvature $Hn$:}

\[\langle \phi, \, H n \rangle = \langle \nabla_\gamma \phi, \, \nabla_\gamma\,
\text{id}_\gamma \rangle \]


% SECTION: Evolution Equations on Stationary Surfaces
\section{Evolution Equations on Stationary Surfaces}


% subsection: Heat Equation
\subsection{Heat Equation}

\begin{equation}
   u_t = \kappa \Delta_{\gamma} u
   \label{eqn:heat}
\end{equation}

{\bf Weak form:} 

{\bf Discretization:} the $\theta$-method: 
\begin{equation}
   \langle \phi_h, \, \frac{u_h^{n+1} - u_h^n}{\tau} \rangle = \langle \phi_h,
   \, \left[ (1 - \theta)\Delta_\gamma u_h^{n} + \theta
   \Delta_{\gamma} u_h^{n+1}\right]\rangle
   \label{disc:heat_theta}
\end{equation}




% subsection: Allen-Cahn 
\subsection{Allen-Cahn}

\begin{equation}
u_t = \eps \Delta_{\gamma} u + \lambda (u - u^3)
   \label{eqn:heat}
\end{equation}




% SECTION: Evoling Surfaces
\section{Evolving Surfaces}

The classical Helfrich model (\ref{eqn:helfrich_energy}) approximates the energy
required to hold a phospholipid bilayer in a geometric configuration that
deviates from its preferred, or spontaneuous, curvature. The model also
includes energy penalties $\lambda$ and
$\mu$ for changes in surface area and enclosed volume. Treating $\lambda$ and
$\mu$ as Lagrange multipliers capture the fact that surface area and enclosed
volume changes happen on a longer timescale than conformational changes.\\



For $\Gamma \subset \R^3$, a closed smooth codimension-one surface
\begin{equation}
   E({\gamma}) := \kappa(x) \int_{\gamma} \left( H(x) - c_{0}(x)\right)^2  + \lambda + \mu\, s
\cdot n \, dS 
   \label{eqn:helfrich_energy}
\end{equation}
where $H(x) = k_1(x) + k_2(x)$: sum of \emph{principle curvatures} $k_1$ and
$k_2$, $c_0(x)$: \emph{spontaneous curvature}, and $\kappa(x)$: \emph{bending
modulus}.


\cite{DoganNochetto:2012}


% SECTION: Evolution Equations on Evoling Surfaces
\section{Evolution Equations on Evolving Surfaces}



% SECTION: Numerics
\section{Numerics}

\subsection{Schur Complement}

\subsection{adaptive time stepping}

\subsection{adaptive mesh refinement/coarsening}





\bibliographystyle{plain}
\bibliography{../../../vedas/archives/references/runningBib}


\end{document}
%EOF
