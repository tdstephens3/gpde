% vector_mean_curvature_computation.tex   Sept 18, 2016

\documentclass[10pt]{article}
\usepackage{amsmath}
\usepackage{amsfonts}
\usepackage{amssymb}
\usepackage{hyperref}
%\usepackage{color}
%\usepackage{graphicx}
%
%\usepackage[top=1.25cm, bottom=2cm, left=2.0cm, right=1.5cm]{geometry} 
%
%\pdfpagewidth 8.5in
%\pdfpageheight 11in
%
\newcommand{\R}{\mathbb{R}}
%\newcommand{\mbf}[1]{\mathbf{#1}}
\newcommand{\mbf}[1]{\boldsymbol{#1}}

\title{Computing vector curvature and total curvature on a triangulated surface}
\author{a  \texttt{deal.ii} implementation}

\date{ }

%%%%%%%%%%%%
\begin{document}         %%% <----------  BEGIN DOCUMENT
%%%%%%%%%%%%
\maketitle


\section*{Introduction}

In this example we show how to compute the vector curvature (sometimes called
\emph{normal curvature}) $\mbf{h}$ on a parametric surface $S$ in $\R^3$.
We include a function to extract the (scalar) total curvature $h$ (sometimes
called \emph{mean curvature}), and finally we test our computations against the
analytic expression for total curvature on ellipsoids.  

We follow the notation of **step 38** and introduce the codimension 1 surface
$S$ which is known by a parameterization $\mbf{x}_{S}: \hat{S} \to S$, where
$\hat{S} \subset \R^2$, and $\mbf{x}_S$ takes points $\hat{\mbf{x}} =
(\hat{x}_1, \hat{x}_2) \in \hat{S}$ to $S$.  In what follows we will continue
to use the cumbersome notation $\mbf{x}_S$ to denote points in $\R^3$ that lie
on $S$, and $\mbf{x}$ to denote arbitrary points in $\R^3$.  The purpose of
this is to emphasize \dots 

The total curvature $h$ on $S$ is defined at every point
$\mbf{x}_S \in S$ and can be described as the sum of the principle curvatures
$\kappa_1(\mbf{x}_S) + \kappa_2(\mbf{x}_S)$ of $S$, hence 
\[h(\mbf{x}_S) =  \kappa_1(\mbf{x}_S) + \kappa_2(\mbf{x}_S).\]
Another way to think about the total curvature in a geometric sense is as the
tangential divergence of the normal vector, that is $h = \text{div}_{S}
\mbf{n} \overset{\text{id}}{=} \nabla_S \cdot \mbf{n}$  (this will appear again below).

The vector curvature $\mbf{h}$ on $S$ is simply defined as the (scalar) total
curvature times the normal vector, hence \[\mbf{h}(\mbf{x}_S) =
h(\mbf{x}_S)\mbf{n}(\mbf{x}_S).\]   It is the vector curvature that we will be
computing first because, quite unlike the (scalar) total curvature, $\mbf{h}$
has an emminently useful formulation in terms of the identity $\text{\bf id}_S
: \mbf{x}_S \mapsto \mbf{x}_S$, and the Laplace Beltrami operator $\Delta_S$ on
$S$.  In fact, 
\[ - \Delta_S \, \text{\bf id}_S = \mbf{h}.\]

There is more wrapped up in the notation above than first meets the eye.
First, $\text{\bf id}_S$ is vector-valued, so the Laplace-Beltrami
operator here is meant to be taken componentwise (and will be made explicit
below). Second, recall that we have defined $S$ through the parameterization
$\mbf{x}_S : (\hat{x}_1, \hat{x}_2) \to S \subset \R^3$.  This makes $\text{\bf
id}_S$ appear as the funtion $\text{\bf id}_S :=  \left( \text{\bf id} \circ
\mbf{x}_S\right) (\hat{x}_1,\hat{x}_2) = \mbf{x}_S(\hat{x}_1,\hat{x}_2)$ for
$(\hat{x}_1,\hat{x}_2) \in \hat{S} \subset \R^2$.  In this setting, the
derivatives involved in $\Delta_S$ must be taken back in $\hat{S}$, which is
cumbersome and best left for \texttt{deal.ii} to do for us (cf **step 38**), yet we want to
unpack at least some of what has been written above before we proceed with the
program.  To this end, we hypothesize an extension of $\text{\bf id}_S$ to all
of $\R^3$ (which is absolutely trivial in this case!), and take our derivatives in
$\R^3$. 

We now quickly show that it is reasonable that we may get the vector curvature
from the surface Laplacian of the identity: Let $\text{\bf id} : \R^3 \to \R^3$
extend the identity $\text{\bf id}_S$ to at least a neighborhood of $S$ in $\R^3$, and write $\text{\bf
id}(\mbf{x}) = \left( \text{id}_1(\mbf{x}), \text{id}_2(\mbf{x}),\text{id}_3(\mbf{x}) \right) = \left( x_1, x_2, x_3 \right)$. Then generalizing the
expression for the Laplace Beltrami operator in **step 38** to vector-valued
functions, we write
\[
 \Delta_S \, \text{\bf id}_S :=
  \left( \begin{array}{c} 
     \Delta \text{id}_1  - \mbf{n}^T D^2 \text{id}_1 \, \mbf{n} - (\mbf{n} \cdot \nabla \text{id}_1)\left( \nabla \cdot \mbf{n} - \mbf{n}^T D\mbf{n} \, \mbf{n} \right) \\ 
     \Delta \text{id}_2  - \mbf{n}^T D^2 \text{id}_2 \, \mbf{n} - (\mbf{n} \cdot \nabla \text{id}_2)\left( \nabla \cdot \mbf{n} - \mbf{n}^T D\mbf{n} \, \mbf{n} \right) \\ 
     \Delta \text{id}_3  - \mbf{n}^T D^2 \text{id}_3 \, \mbf{n} - (\mbf{n} \cdot \nabla \text{id}_3)\left( \nabla \cdot \mbf{n} - \mbf{n}^T D\mbf{n} \, \mbf{n} \right) 
  \end{array} \right). 
\]
First notice that $\left( \nabla \cdot \mbf{n} - \mbf{n}^T D\mbf{n} \, \mbf{n}
\right) = \text{div}_{S} \mbf{n}$, which we claimed to be $h(\mbf{x}_S)$ above
(see **step 38** and \cite{} for more about tangential derivatives).  Since
we're dealing with identity functions all around, we have that $\nabla
\text{id}_i = \mbf{e}_i$,  $\Delta \text{id}_i = 0,$ and
$D^2 \text{id}_i$ is the zero matrix for $i=1,2,3$, leaving us with 
\[
 \Delta_S \, \text{\bf id}_S =
  \left( \begin{array}{c} 
     - (\mbf{n} \cdot \left( 1,0,0 \right)) h(\mbf{x}_S)  \\ 
     - (\mbf{n} \cdot \left( 0,1,0 \right)) h(\mbf{x}_S)  \\ 
     - (\mbf{n} \cdot \left( 0,0,1 \right)) h(\mbf{x}_S) 
  \end{array} \right) \\
  = 
  \left( \begin{array}{c} 
     - h(\mbf{x}_S) n_1  \\ 
     - h(\mbf{x}_S) n_2  \\ 
     - h(\mbf{x}_S) n_3 
  \end{array} \right) 
  = - \mbf{h}(\mbf{x}_S).
\]

\section*{Strategy For Computations}

Presumably the purpose of computing curvature on a surface is to then use it in
further computations.  This means that we should make sure the data structure that holds
curvature information is be compatible with further finite element comptuations
in \texttt{deal.ii}.  All this is to say that we will use \texttt{deali.ii} to
`solve' $- \Delta_S \, \text{\bf id}_S = \mbf{h}$ for $\mbf{h}$ (more
precisely, we will weakly enforce the identity $\nabla_S \boldsymbol{\varphi}
\nabla_S \text{\bf id} \overset{\text{id}}{=} \mbf{\varphi}\mbf{h}$).  For this
all we really need is a decent mesh of points lying on $S$ so that we have a
discrete representation of $\text{\bf id}_S$.  Instead of only specifying the
surface $S$ by vertices of a mesh, we will instead specify a
\texttt{MappingQEulerian<dim,spacedim> mapping(\dots)} object initialized with a callable
\texttt{Function<spacedim> map\_to\_S(spacedim)} which parameterizes $S$.  This
allows us to use higher-order finite elements because we can specify $S$ at
**support points** as well as at the vertices, enhancing our discrete
representation of $\text{\bf id}_S$ and thereby enhancing the accuracy of our
approximation of $\mbf{h}$. 

As noted in **step-38**, the user will not be expressing derivatives back in
$\hat{S}$.  Rather, the user will specify a \texttt{Mapping} on a
\texttt{Triangulation} (or specifying just a \texttt{Triangulation}, and
letting \texttt{deal.ii} associate a default piecewise-linear \texttt{Mapping}
on that \texttt{Triangulation})


\section*{Test Case}

Ellipsoid parameterized, $\hat{S} = [0,2\pi] \times [0, \pi]$ by 
\[  \mbf{x}(\hat{x}_1, \hat{x}_2) = 
   \left( \begin{array}{c} 
      a\sin \hat{x}_1 \cos \hat{x}_2 \\
      b\sin \hat{x}_1 \sin \hat{x}_2 \\
            c\cos\hat{x}_1 
   \end{array} \right)
\]
has total (scalar) curvature 
\begin{align*}
   h(\hat{x}_1, \hat{x}_2) &= \frac{2abc\left[ 3(a^2 + b^2) + 2c^2 + (a^2 + b^2 -
   2c^2)\cos(2\hat{x}_1) - 2(a^2 - b^2)\cos(2\hat{x}_2)\sin^2 \hat{x}_1 \right]}
                           {  8\left[a^2 b^2 \cos^2 \hat{x}_1 + c^2(b^2\cos^2 \hat{x}_2 +
                           a^2\sin^2 \hat{x}_2 ) \sin^2 \hat{x}_1 \right]^{3/2}}
\end{align*}

\begin{verbatim}
template<int spacedim>
double ExactScalarMeanCurvatureOnEllipsoid<spacedim>::value(const Point<spacedim> &p, const unsigned int )  const
{
  Point<spacedim> unmapped_p(p(0)/a, p(1)/b,  p(2)/c);

  Point<spacedim> chart_point = spherical_manifold.pull_back(unmapped_p);
  // double radius = chart_point(0); 
  double x1_hat = chart_point(1);
  double x2_hat = chart_point(2);
  ...  
}
\end{verbatim}


We don't reimplement the parameterization from $\hat{S}$ to $\R^3$ here -- instead we
use the built-in \texttt{SphericalManfold<dim,spacedim> sphere} and rescale
\texttt{sphere} in $\R^3$.  The
\texttt{pull\_back} and \texttt{push\_forward} functions allow us to write our
functions in terms of $\mbf{x} \in \R^3$ rather than $\hat{\mbf{x}} \in \hat{S}
\subset \R^2$. 

\begin{verbatim}
template<int spacedim>
Tensor<1,spacedim> ExactVectorMeanCurvatureOnEllipsoid<spacedim>::value(const Point<spacedim> &p)  const
{
  Tensor<1,spacedim> normal,vector_H;
  normal[0] = p(0)/a;
  normal[1] = p(1)/b;
  normal[2] = p(2)/c;
  normal /= normal.norm();

  vector_H  = normal;
  vector_H *= exact_scalar_H.value(p);

  return vector_H;
}
\end{verbatim}

\section*{Implementation}



\end{document}
